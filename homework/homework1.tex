\documentclass[letter]{article}
\usepackage{amsmath}
\usepackage{amsfonts}
\usepackage{amssymb}
\usepackage{bm}
\usepackage{ifthen}
\usepackage{fancyhdr}
\usepackage[hidelinks]{hyperref}
\usepackage{xcolor}
\hypersetup{
    colorlinks,
    linkcolor={red!50!black},
    citecolor={blue!50!black},
    urlcolor={blue!80!black}
}

\definecolor{deepblue}{rgb}{0,0,0.5}
\definecolor{deepred}{rgb}{0.6,0,0}
\definecolor{deepgreen}{rgb}{0,0.5,0}

\usepackage{listings}

% Default fixed font does not support bold face
\DeclareFixedFont{\ttb}{T1}{txtt}{bx}{n}{12} % for bold
\DeclareFixedFont{\ttm}{T1}{txtt}{m}{n}{12}  % for normal

% Python style for highlighting
\newcommand\pythonstyle{\lstset{
language=Python,
basicstyle=\ttm,
otherkeywords={self},             % Add keywords here
keywordstyle=\ttb\color{deepblue},
emph={MyClass,__init__},          % Custom highlighting
emphstyle=\ttb\color{deepred},    % Custom highlighting style
stringstyle=\color{deepgreen},
frame=tb,                         % Any extra options here
showstringspaces=false            % 
}}


% Python environment
\lstnewenvironment{python}[1][]
{
\pythonstyle
\lstset{#1}
}
{}

% Python for external files
\newcommand\pythonexternal[2][]{{
\pythonstyle
\lstinputlisting[#1]{#2}}}

% Python for inline
\newcommand\pythoninline[1]{{\pythonstyle\lstinline!#1!}}


%%%
% Set up the margins to use a fairly large area of the page
%%%
\oddsidemargin=.2in
\evensidemargin=.2in
\textwidth=6in
\topmargin=0in
\textheight=9.0in
\parskip=.07in
\parindent=0in
\pagestyle{fancy}

%%%
% Set up the header
%%%
\newcommand{\setheader}[6]{
	\lhead{{\sc #1}\\{\sc #2} ({\small \it \today})}
	\rhead{
		{\bf #3} 
		\ifthenelse{\equal{#4}{}}{}{(#4)}\\
		{\bf #5} 
		\ifthenelse{\equal{#6}{}}{}{(#6)}%
	}
}

\makeatletter
\newcommand{\escapeus}{\begingroup\@makeother\_\@escapeus}
\newcommand*{\@escapeus}[1]{#1\endgroup}
\makeatother

%%%
% Set up some shortcut commands
%%%
\newcommand{\R}{\mathbb{R}}
\newcommand{\N}{\mathbb{N}}
\newcommand{\Z}{\mathbb{Z}}
\newcommand{\Proj}{\mathrm{proj}}
\newcommand{\Perp}{\mathrm{perp}}
\newcommand{\proj}{\mathrm{proj}}
\newcommand{\Span}{\mathrm{span}}
\newcommand{\Null}{\mathrm{null}}
\newcommand{\Rank}{\mathrm{rank}}
\newcommand{\mat}[1]{\begin{bmatrix}#1\end{bmatrix}}
\newcommand{\var}[1]{{$\langle$\it #1$\rangle$}}
\newcommand{\Code}[1]{\texttt{\escapeus #1}}

%%%
% This is where the body of the document goes
%%%
\begin{document}
	\setheader{MAT335}{Homework 1}{Due: 11:59pm January 26}{}{}{}

	\begin{enumerate}
		\item Let $(T,X)$ be a dynamical system.
			\begin{enumerate}
				\item Prove that if $x\in X$ is a fixed point of $T$, then 
					the \emph{basin of attraction} of $x$, $A_x$, is non-empty.
				\item Fix $y\in X$. Prove that if $R\subseteq A_y$, then $T^{-1}(R)\subseteq A_y$.
					(Recall $T^{-1}(R) = \{w\in X: T(w)\in R\}$ is the \emph{inverse image
					of $R$ under $T$}. $T$ is {\bfseries not necessarily} invertible.)
			\end{enumerate}
		\item You will prove that under certain conditions, Newton's method converges.

		Let $f:\R\to\R$ be a twice differentiable function and suppose that $f(0)=0$ and that
			$f'$ and $f''$ are positive on the interval $(0,a]$. Let $T_f$ be the function
			that applies one iteration of Newton's method to its input.
		\begin{enumerate}
			\item Find a function $f$ satisfying the required properties and graph it. Does it look
				like Newton's method will converge on $(0,a]$?

			\item Show that if $x\in(0,a)$, then $0\leq T_f(x)< a$.

			\item The \emph{Monotone Convergence Theorem} states that if $(a_n)$ is a sequence
				of real numbers that is bounded below \emph{and} $a_{n+1}\leq a_n$, then 
				$\displaystyle \lim_{n\to\infty} a_n$ exists and is a real number.

				Use the Monotone Convergence Theorem to prove that $\displaystyle \lim_{n\to\infty}T^n_f(x)$
				converges for all $x\in(0,a]$.
			\item A point $y$ is called a \emph{fixed point} of a function $g$ if $g(y)=y$.

				Fix $x_0\in(0,a]$, and define $\displaystyle \bm x=\lim_{n\to\infty} T_f^n(x_0)$. Show
				that $\bm x$ is a fixed point of $T_f$.

				\emph{Hint: You may use the fact that if $g$ is a continuous function, then
				$g(\lim_{n\to\infty} a_n) = \lim_{n\to\infty} g(a_n)$ for any convergent sequence $(a_n)$.}
			\item Prove that for any $x\in(0,a]$, $\displaystyle \lim_{n\to\infty} T^n_f(x)=0$.
			\item Combine your results and explain why Newton's method will always converge for $f$ if
				you pick an initial guess in $(0,a]$.
		\end{enumerate}

		\item Let's prove more about Newton's method! Suppose $f:\R\to\R$ is twice differentiable and
			$f(0)=0$. Further, suppose $f'$ and $f''$ are both positive on the interval $[-a,a]$ (for
			some $a>0$). 
			\begin{enumerate}
				\item Use a picture to argue that Newton's method might not converge for some $x\in[-a,a]$.
				\item Show that there is some $b>0$ so that for $x\in[-b,a]$, Newton's method always converges.
				\item Let $g:\R\to\R$ and define $h:\R\to\R$ by $h(t)=g(-t)$. Prove that if Newton's method
					converges for $g$ with a starting point of $x_0$, then Newton's method converges
					for $h$ with a starting point of $-x_0$.
				\item Prove that if $f:\R\to\R$ satisfies $f(0)=0$, and $f' < 0$ and $f'' > 0$ on the interval
					$[-a,a]$, then there exists a $b>0$ so that Newton's method converges for
					$f$ on $[-a,b]$.
			\end{enumerate}

		\item Let $f:\R\to\R$ be defined by $f(x)=x^2-1$ and let $T_f$ be the function that applies Newton's method
			to its inputs.
			\begin{enumerate}
				\item Find $T_f^{-1}(\{1,2,3\})$ (recall that $T_f^{-1}(\{1,2,3\})$ is the \emph{inverse image}
					of the set $\{1,2,3\}$ under $T_f$. It is not the same as applying the \emph{inverse}
					of $T_f$, since $T_f$ might not be invertible).
				\item Is $T_f(x)$ defined for all $x\in \R$?
				\item Find the largest possible domain, $X\subseteq \R$, so that $T_f:X\to X$ is a \emph{dynamical
					system}.
				\item Prove that $\lim_{n\to\infty} T^n_f(4)$ exists. What value is it?
				\item Find all fixed points of $T_f$.
				\item For each fixed point of $T_f$, find its \emph{basin of attraction}.
			\end{enumerate}

		\item Let $f:\R\to\R$ be defined by $f(x)=\frac{2}{1+x^2}-\frac{5}{4}$ and let $T_f$ be the function that applies Newton's method
			to its inputs.
			\begin{enumerate}
				\item Is $0$ in the domain of $T_f$?
				\item Find $T_f^{-1}(\{0\})$ (you can use a computer algebra system).
				\item Find $T_f^{-2}(\{0\})$ (you can use a computer algebra system).
				\item Describe largest possible domain, $X\subseteq \R$, so that $T_f:X\to X$ is a \emph{dynamical
					system}. How many connected components does this domain consist of? (Use computers to help you!)
			\end{enumerate}
	\end{enumerate}

\end{document}
