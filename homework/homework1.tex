\documentclass[letter]{article}
\usepackage{amsmath}
\usepackage{amsfonts}
\usepackage{amssymb}
\usepackage{bm}
\usepackage{ifthen}
\usepackage{fancyhdr}
\usepackage{graphicx}
\usepackage[hidelinks]{hyperref}
\usepackage{xcolor}
\hypersetup{
    colorlinks,
    linkcolor={red!50!black},
    citecolor={blue!50!black},
    urlcolor={blue!80!black}
}

\definecolor{deepblue}{rgb}{0,0,0.5}
\definecolor{deepred}{rgb}{0.6,0,0}
\definecolor{deepgreen}{rgb}{0,0.5,0}

\usepackage{listings}

% Default fixed font does not support bold face
\DeclareFixedFont{\ttb}{T1}{txtt}{bx}{n}{12} % for bold
\DeclareFixedFont{\ttm}{T1}{txtt}{m}{n}{12}  % for normal

% Python style for highlighting
\newcommand\pythonstyle{\lstset{
language=Python,
basicstyle=\ttm,
otherkeywords={self},             % Add keywords here
keywordstyle=\ttb\color{deepblue},
emph={MyClass,__init__},          % Custom highlighting
emphstyle=\ttb\color{deepred},    % Custom highlighting style
stringstyle=\color{deepgreen},
frame=tb,                         % Any extra options here
showstringspaces=false            % 
}}


% Python environment
\lstnewenvironment{python}[1][]
{
\pythonstyle
\lstset{#1}
}
{}

% Python for external files
\newcommand\pythonexternal[2][]{{
\pythonstyle
\lstinputlisting[#1]{#2}}}

% Python for inline
\newcommand\pythoninline[1]{{\pythonstyle\lstinline!#1!}}


%%%
% Set up the margins to use a fairly large area of the page
%%%
\oddsidemargin=.2in
\evensidemargin=.2in
\textwidth=6in
\topmargin=-.5in
\textheight=9in
\parskip=.07in
\parindent=0in
\pagestyle{fancy}

%%%
% Set up the header
%%%
\newcommand{\setheader}[6]{
	\lhead{{\sc #1}\\{\sc #2} ({\small \it \today})}
	\rhead{
		{\bf #3} 
		\ifthenelse{\equal{#4}{}}{}{(#4)}\\
		{\bf #5} 
		\ifthenelse{\equal{#6}{}}{}{(#6)}%
	}
}

\makeatletter
\newcommand{\escapeus}{\begingroup\@makeother\_\@escapeus}
\newcommand*{\@escapeus}[1]{#1\endgroup}
\makeatother

%%%
% Set up some shortcut commands
%%%
\newcommand{\R}{\mathbb{R}}
\newcommand{\N}{\mathbb{N}}
\newcommand{\Z}{\mathbb{Z}}
\newcommand{\Proj}{\mathrm{proj}}
\newcommand{\Perp}{\mathrm{perp}}
\newcommand{\proj}{\mathrm{proj}}
\newcommand{\Span}{\mathrm{span}}
\newcommand{\Null}{\mathrm{null}}
\newcommand{\Rank}{\mathrm{rank}}
\newcommand{\mat}[1]{\begin{bmatrix}#1\end{bmatrix}}
\newcommand{\var}[1]{{$\langle$\it #1$\rangle$}}
\newcommand{\Code}[1]{\texttt{\escapeus #1}}

%%%
% This is where the body of the document goes
%%%
\begin{document}
	\setheader{MAT335}{Homework 1}{Due: 11:59pm January 26}{}{}{}

	\begin{enumerate}
		\item Let $(T,X)$ be a dynamical system.
			\begin{enumerate}
				\item Prove that if $x\in X$ is a fixed point of $T$, then 
					the \emph{basin of attraction} of $x$, $A_x$, is non-empty.
				\item Fix $y\in X$. Prove that if $R\subseteq A_y$, then $T^{-1}(R)\subseteq A_y$.
					(Recall $T^{-1}(R) = \{w\in X: T(w)\in R\}$ is the \emph{inverse image
					of $R$ under $T$}. $T$ is {\bfseries not necessarily} invertible.)
			\end{enumerate}
		\item You will prove that under certain conditions, Newton's method converges.

		Let $f:\R\to\R$ be a twice differentiable function and suppose that $f(0)=0$ and that
			$f'$ and $f''$ are positive on the interval $(0,a]$. Let $T_f$ be the function
			that applies one iteration of Newton's method to its input.
		\begin{enumerate}
			\item Find a function $f$ satisfying the required properties and graph it. Does it look
				like Newton's method will converge on $(0,a]$?

			\item Show that if $x\in(0,a)$, then $0\leq T_f(x)< a$.

			\item The \emph{Monotone Convergence Theorem} states that if $(a_n)$ is a sequence
				of real numbers that is bounded below \emph{and} $a_{n+1}\leq a_n$, then 
				$\displaystyle \lim_{n\to\infty} a_n$ exists and is a real number.

				Use the Monotone Convergence Theorem to prove that $\displaystyle \lim_{n\to\infty}T^n_f(x)$
				converges for all $x\in(0,a]$.
			\item A point $y$ is called a \emph{fixed point} of a function $g$ if $g(y)=y$.

				Fix $x_0\in(0,a]$, and define $\displaystyle \bm x=\lim_{n\to\infty} T_f^n(x_0)$. Show
				that $\bm x$ is a fixed point of $T_f$.

				\emph{Hint: You may use the fact that if $g$ is a continuous function, then
				$g(\lim_{n\to\infty} a_n) = \lim_{n\to\infty} g(a_n)$ for any convergent sequence $(a_n)$.}
			\item Prove that for any $x\in(0,a]$, $\displaystyle \lim_{n\to\infty} T^n_f(x)=0$.
			\item Combine your results and explain why Newton's method will always converge for $f$ if
				you pick an initial guess in $(0,a]$.
		\end{enumerate}

		\item Let's prove more about Newton's method! Suppose $f:\R\to\R$ is twice differentiable and
			$f(0)=0$. Further, suppose $f'$ and $f''$ are both positive on the interval $[-a,a]$ (for
			some $a>0$). 
			\begin{enumerate}
				\item Use a picture to argue that Newton's method might not converge for some $x\in[-a,a]$.
				\item Show that there is some $b>0$ so that for $x\in[-b,a]$, Newton's method always converges.
				\item Let $g:\R\to\R$ and define $h:\R\to\R$ by $h(t)=g(-t)$. Prove that if Newton's method
					converges for $g$ with a starting point of $x_0$, then Newton's method converges
					for $h$ with a starting point of $-x_0$.
				\item Prove that if $f:\R\to\R$ satisfies $f(0)=0$, and $f' < 0$ and $f'' > 0$ on the interval
					$[-a,a]$, then there exists a $b>0$ so that Newton's method converges for
					$f$ on $[-a,b]$.
				\item Prove that if $p:\R\to\R$ is a twice differentiable function satisfying $p(x_0)=0$ and 
					$p'(x_0),p''(x_0)\neq 0$, then there is an open interval about $x_0$ on which Newton's
					method will converge to $x_0$.
			\end{enumerate}

		\item Let $f:\R\to\R$ be defined by $f(x)=x^2-1$ and let $T_f$ be the function that applies Newton's method
			to its inputs.
			\begin{enumerate}
				\item Find $T_f^{-1}(\{1,2,3\})$ (recall that $T_f^{-1}(\{1,2,3\})$ is the \emph{inverse image}
					of the set $\{1,2,3\}$ under $T_f$. It is not the same as applying the \emph{inverse}
					of $T_f$, since $T_f$ might not be invertible).
				\item Is $T_f(x)$ defined for all $x\in \R$?
				\item Find the largest possible domain, $X\subseteq \R$, so that $T_f:X\to X$ is a \emph{dynamical
					system}.
				\item Prove that $\lim_{n\to\infty} T^n_f(4)$ exists. What value is it?
				\item Find all fixed points of $T_f$.
				\item For each fixed point of $T_f$, find its \emph{basin of attraction}.
			\end{enumerate}

		\item Let $f:\R\to\R$ be defined by $f(x)=\frac{2}{1+x^2}-\frac{5}{4}$ and let $T_f$ be the function that applies Newton's method
			to its inputs.
			\begin{enumerate}
				\item Is $0$ in the domain of $T_f$?
				\item Find $T_f^{-1}(\{0\})$ (you can use a computer algebra system).
				\item Find $T_f^{-2}(\{0\})$ (you can use a computer algebra system).
				\item Describe largest possible domain, $X\subseteq \R$, so that $T_f:X\to X$ is a \emph{dynamical
					system}. How many connected components does this domain consist of? (Use computers to help you!)
			\end{enumerate}

		\item Recall the \emph{box-counting dimension} from the course notes. Throughout this problem, if we
			refer to the box-counting dimension of a set, you may assume that the box-counting dimension of that
			set exists (i.e., the limit in the definition converges to a finite number).
			\begin{enumerate}
				\item Prove that the line segment from $\vec 0$ to $\sum \vec e_i$ in $\R^d$ has box-counting dimension $1$.
				\item In the definition of the box-counting dimension of $X\subseteq \R^m$, 
					the set $B$ is taken to be a minimal-dimensional, minimally-sized 
					superset of $X$. Show that we can drop the both the
					assumption that $B$ is minimally sized and that $B$ is of a minimal dimension. (\emph{Hint: 
					prove these separately.})
				\item Prove that box-counting dimension is \emph{translation invariant}. That is, if $Y$ is
					a translation of $X\subseteq\R^m$, then $X$ and $Y$ have the same box-counting dimension.
				\item Sets $X,Y\subseteq \R^m$ are called \emph{box-disjoint} if there exists disjoint $m$-dimensional
					boxes $A,B$ so that $X\subseteq A$ and $Y\subseteq B$.
					Prove that if $X,Y\subseteq \R^m$ are box-disjoint sets, then 
					\[
						\dim(X\cup Y) = \max\{\dim(X),\dim(Y)\},
					\]
					where $\dim$ stands for the box-counting dimension.
				\item Show that
					if $T:\R^m\to\R^m$ is a dilation given by $\vec x\mapsto k\vec x$ for some $k\in\Z^+$ 
					and $X\subseteq \R^m$ is a set, then
					the box-counting dimension of $X$ equals the box-counting dimension of $T(X)$.
				\item Show that if $T:\R^m\to\R^m$ is a linear transformation, the box-counting dimension of
					$X\subseteq \R^m$ is bounded above by the box-counting dimension of $T(X)$.
				\item Show that if $T:\R^m\to\R^m$ is an invertible linear transformation, then the box-counting
					dimension of $X\subseteq\R^m$ and $T(X)$ are equal.
				\item (Optional) Show that if $\varphi:\R^m\to\R^m$ is a differentiable function
					with continuous derivative
					and $\det(D\varphi)\neq 0$, then the box-counting dimension of $X\subseteq \R^m$ and
					$\varphi(X)$ are equal.
			\end{enumerate}


	\end{enumerate}


	\subsection*{Programming Problems}
	For the programming problems, please use the Jupyter notebook available at

	\url{https://utoronto.syzygy.ca/jupyter/user-redirect/git-pull?repo=https://github.com/siefkenj/2020-MAT-335-webpage&subPath=homework/homework1-exercises.ipynb}

	Make sure to comment your code and use ``Markdown'' style cells to explain what your answers.

	\begin{enumerate}
		\item Implement Newton's method for the function $f(x)=x(x-2)(x-3)$.
			\begin{enumerate}
				\item Functions \verb|f| and \verb|f_prime| are already provided. Implement the functions
					\verb|T_f| and \verb|newt|. \verb|T_f| inputs a guess and applies one iteration
					of Newton's method. \verb|newt| should repeatedly apply Newton's method until a root
					is reached.

					Since Newton's method may take an infinite number of iterations to converge, we 
					won't require a point to actually reach a root. Instead, \verb|newt| will input a 
					\emph{tolerance}. If $|f(x)|<\text{\emph{tolerance}}$, we will say that $x$ is 
					close enough to a root, and we won't apply Newton's method any more.

					However, we also know Newton's method can totally fail to converge! To prevent this
					from causing an issue,
					\verb|newt| also accepts a variable \verb|max_iterations|. Your \verb|newt| function
					should apply \verb|T_f| at most \verb|max_iterations| number of times. If it hasn't found
					a root by then, return \verb|np.nan|\footnote{ {\tt np.nan} stands for \emph{Not a Number}
					and is a way to indicate there is no numeric solution in Python.}.
				\item \verb|v_newt| is a \emph{vectorized} version of \verb|newt|, which means that when given an
					array as input, it applies \verb|newt| to each element of the array. Execute the next cell
					to plot the result of Newton's method and $f$.
				\item Plot a version of the above graph that is ``zoomed-in'' to the boundary between two basins of attraction.
				\item Make a plot showing just the basin of attraction of $3$ and the graph of $f$.
			\end{enumerate}
		\item Making Newton's method complex!
			Complex numbers can be thought of as two-dimensional analogs of real numbers---they can be
			added, multiplied, and divided. But, a complex number has two ``coordinates'' called the \emph{real part}
			and the \emph{imaginary part}. Oftentimes we visualize complex numbers by plotting them in $\R^2$
			with their real part along the $x$-axis and imaginary part along the $y$-axis.

			Since Newton's method involves only basic operations that also work on complex numbers,
			we can apply it to complex values and see what various basins of attraction look like as subsets of
			the complex plane.
			\begin{enumerate}
				\item Since we are now dealing with 2d arrays which have \emph{way} more points in them,
					it will be convenient to make a fast-and-cheap version of Newton's method. Numpy
					has many pre-vectorized built-in functions that work very fast. Among these
					\var{array}\verb|*|\var{array}, \var{array}\verb|/|\var{array}, \var{number}\pm\var{array}
					all perform very fast element-wise operations. This means the \verb|T_f| you created before
					is already vectorized!

					Create a \verb|newt2| function which inputs an array and a number of iterations
					and applies \verb|T_f| to the input array the specified number of times\footnote{ Your original
					function would stop iterating when you were sufficiently close to the root. This one will not,
					but because of how fast Numpy functions are compared to Python functions, it will actually
					be faster.}.
				\item The \verb|newt2| function indiscriminately apples \verb|T_f|, and after it executes, some points may
					have converged and some may not have. However, since we know the roots of $f$, we 
					can nudge our output
					to ensure it's correct.

					Create a function \verb|clamped_newt| which applies \verb|newt2| to its input 
					(with the specified number of iterations) and then returns values that have been ``clamped''
					to $0$, $2$, or $3$. That is, every output value in the resulting array should be $0$, $2$, or $3$,
					depending on which number it is closest to.

				\item Plot the basins of attractions of $f$ in the complex plane in the rectangle with lower-left corner
					$-1/2-i$ and upper-right corner $7/2+i$. 

					\emph{Notes:}
					\begin{itemize}
						\item Python uses {\tt j} for imaginary numbers
							instead of $i$, but you can't use `{\tt j}' alone. For example, to get the number $i$, you must
							write {\tt 1j} in Python\footnote{ Since {\tt i} is so commonly used
							as a loop variable in programming, most programming languages use {\tt j}
							for complex numbers.}.
						\item If you get a {\tt TypeError: Image data cannot be converted to float}
							error, you're probably asking {\tt imshow} to graph a complex image. {\tt imshow}
							only knows how to graph
					real images; you might try calling {\tt np.real(...)} on your data first.
					\end{itemize}

				\item Make a new plot that is zoomed into an interesting region.
			\end{enumerate}
		\item Investigate another Newton fractal of your choosing. Some interesting functions you might try: $f(x)=sin(x^2)-1$,
			$f(x)=x^3-1$, $f(x)=(\tfrac{2}{1+x^2}-\tfrac{5}{4})(1+x^{1.5})$. Produce two plots of your Newton fractal, one zoomed
			in and one zoomed out.

			Some things to keep in mind: \begin{itemize}
				\item You can redefine \verb|f| and \verb|f_prime| and \verb|newt2| will just work.
			However \verb|clamped_newt| will no longer give you the information you want, since the roots have changed. 
				\item {\tt imshow} can only plot reals, so use {\tt np.real} and {\tt np.imag} appropriately.
				\item Wild functions tend to have wild values. If you see a solid color, there may still be a fractal buried in there,
				but it is being washed out by very large numbers in your array. Try eliminating large (and large negative) numbers.
				\item If you want to use non-rational functions, use their Numpy versions. E.g., {\tt np.sin} and {\tt np.exp}.
					That way they will be automatically vectorized.
				\item (Optional) If you want to learn another package, {\tt sympy} can symbolically represent math in Python (and
					compute derivatives for you). {\tt sympy.utilities.lambdify} can be used to convert a Sympy
					function into a Numpy function. Google for documentation to see some examples.
			\end{itemize}
		\item {\bf The Cantor set} 
			\begin{enumerate}
				\item The Jupyter notebook includes a \verb|cantorize| function which takes in a list
				of line segments and removes the middle third from each. Review and execute this function and its associated cells
				(which will produce a picture of the Cantor set).
				\item We can use Numpy to estimate the number of ``boxes'' a sequence of line segments intersects.
					The \verb|render_segments_to_array| function inputs a list of line segments, an array,
					and an \verb|extent|, and ``draws'' the line segments to the array by filling in $1$s
					in every coordinate of the array the line segments touch.

					Using this function (and other tools at your disposal),
					numerically estimate the box-counting dimension of the Cantor set.
			\end{enumerate}
		\item {\bf The Koch Snowflake} 
			\begin{enumerate}
				\item Implement a \verb|kochize| function that inputs a list of segments and applies the Koch snowflake
					substitution to each of them. Graph one side of $K_4$ (the Koch snowflake after 4 substitutions).

					For reference, \verb|kochize| applied to {\tt [((0, 0), (1, 0))]} should produce
					{\tt  [((0, 0), (0.3333333333333333, 0.0)), ((0.3333333333333333, 0.0), (0.5, 0.28867513459481287)), ((0.5, 0.28867513459481287), (0.6666666666666666, 0.0)), ((0.6666666666666666, 0.0), (1, 0))]}.
				\item Make a plot of the full Koch snowflake (all sides of it, not just the top side).
				\item Estimate the box-counting dimension of the Koch snowflake. How does this compare with
					its similarity dimension?
			\end{enumerate}
		\item {\bf The Strange Koch Snowflake} 
			\begin{enumerate}
				\item Implement a \verb|strange_kochize| function that inputs a list of segments and applies the Koch snowflake
					substitution to each of them except that it randomly chooses (with 50/50 probability) 
					to point the ``spike'' up or
					down along each segment. Graph the 6$^\text{th}$ iteration of the strange Koch snowflake.
					
					\emph{Notes:}
					\begin{itemize}
						\item In Numpy, you can use \verb|np.random.uniform()| to generate
					a uniform random number in the interval $[0,1]$. So, to execute a statement 50\% of the time,
					you could say, \verb|if np.random.uniform() < .5: ...|.

						\item Every time you re-run your Python code, new random numbers will be generated (and your picture will
					change). To ``fix'' the random numbers, run \verb|np.random.seed(10)| before you run your code (10 can
					be replaced with any number of your choosing).
					\end{itemize}

				\item Estimate the box-counting dimension of the strange Koch snowflake. If you changed the probability of 
					a spike pointing ``up'' to 75/25, would this change the dimension of the resulting fractal?
					Support your claim with evidence.
			\end{enumerate}
		\item (Optional) {\bf Sierpinski's Triangle}. View the \verb|matplotlib.collections| documentation and figure out
			how to draw polygons. Then make a \verb|sierpinskinate| function which applies the Sierpinski triangle
			substitution. Estimate the box-counting dimension of the Sierpinski triangle.
		\item {\bf Our favorite $\bm f$}. Recall $f(x)=x(x-2)(x-3)$.
			\begin{enumerate}
				\item The function \verb|find_edges| will input a 2d array and output an array of 0s and 1s depending
					on whether the input array differs when shifting one index to the right, down, or diagonal. Use
					this function to graph the boundary of the Newton fractal given by $f(x)=x(x-2)(x-3)$.

					\emph{Notes:}
					\begin{itemize}
						\item \verb|find_edges| returns 1 if there is not literal equality of values. For example,
					if one pixel of your image has a value of $3.00001$ and the next has a value of
							$3.00002$, \verb|find_edges| will assume
					there is a boundary there. So: either modify \verb|find_edges| to not be so strict, or make sure
					you have clamped the values in the array that stores your Newton fractal.
					\end{itemize}
				\item Estimate the fractal dimension of the above Newton fractal's boundary.
				\item Graph a Newton fractal whose boundary has higher dimension than the one generated by $f(x)=x(x-2)(x-3)$.
					Back up your claim with evidence.
			\end{enumerate}
	\end{enumerate}



\end{document}
